\section{VEX Reinforcement Learning}
\timestamp{Andrew Needham}{[2/12/25]}{}

\subsection{Goals}
\begin{itemize}
    \item Create a reinforcement learning environment for the VEX robot
    \item Train the robot to use an optimal strategy to score as many points as possible
    \item Update the structure of the environment to better represent the robot and field
\end{itemize}

\subsection{Methods}
\begin{itemize}
    \item Use Gymnasium and Stable Baselines3 to create the environment and train the robot
    \item Use Slurm to run the training script on the cluster to train quickly
\end{itemize}

\subsection{Results}
\begin{itemize}
    \item Added wall stakes to the environment and added actions to move to and ad rings to stakes
    \item Simplified the observations space and added time remaining to the observation space
    \item Fixed some of the restrictions on the actions to account for game rules and physical limitations
    \item Adding a reward for climbing makes the robot reach a local maximum and not explore other actions
    \item Added small chance of actions failing to simulate the robot not being able to perform an action
    \item Experimented with the reward system to give points based on estimated final score at the end of the episode
    \item Noticed that training usually peaked around 2 million steps and slowly started to decline
    \item It seems that after a certain number of steps, the robot starts to overfit the environment and starts repeating the same actions
    \item Updated code to GitHub repo: \href{https://github.com/msoe-vex/VEX-AI-Reinforcement-Learning.git}{https://github.com/msoe-vex/VEX-AI-Reinforcement-Learning.git}
\end{itemize}

\subsection{Action Items}
\begin{itemize}
    \item Update the structure of the environment to better represent the robot and field
    \item Add different colors for rings
    \item Update the environment to account for unknown and missing information about elements on the field
    \item Experiment with the PPO parameters to see if the robot can learn more effectively
\end{itemize}