\section*{LemLib implementation}
\timestamp{Andrew Needham}{[11/06/24]}{}

\subsection*{Research LemLib library}
% Briefly describe the goal of this topic. Try to explain what you explored in the topic you chose.
% Example: The goal of this exploration was to understand the ROS2 environment and how to set up a simulation environment for the GHOST project.

\subsection*{Questions Explored}
% List the specific questions or areas you focused on. You can paste the questions you have in the exploration topic and add any you found during the exploration.
\begin{itemize}
    \item What is LemLib and what does it do?
    \item How can odometry be used to control the robot?
\end{itemize}
% Example:
% \begin{itemize}
%     \item What information can the sim env store?
%     \item What is the filetype of the world and the model files?
%     \item How can we interact with the simulation environment?
% \end{itemize}

\subsection*{Methodology}
% Document the exact steps you followed, including paths to relevant files and commands used. If you tried several approaches, list them as separate lists.
\begin{enumerate}
    \item Reading through LemLib documentation to configure the robot
    \item Use Jerry.io and/or CasADi to plan paths to be used with LemLib
\end{enumerate}
% Example:
% \testbf{Exploration Steps:}
% \begin{enumerate}
%     \item Setup ROS2 environment following the instructions in the GHOST project documentation.
%     \item Launch the simulation environment using the script `launch_sim.sh` in the `scripts` directory.
%     \item Explore the simulation environment in the "default.world" file in the `04_Sim/ghost_sim/worlds` directory and the "models" menu in the Gazebo GUI.
% \end{enumerate}

\subsection*{Code Locations}
% Provide paths to relevant scripts, configuration files, or datasets used. If you used a script, provide the path to the script and the command used to run it.
\begin{itemize}
    \item LemLib source code repository (stable): \href{https://github.com/LemLib/LemLib/tree/stable/}{link}
    \item LemLib documentation: \href{https://lemlib.readthedocs.io/en/stable/}{link}
    \item Test program repo: \href{https://github.com/msoe-vex/CasADi-Test.git}{link}
\end{itemize}
% Example:
% \begin{itemize}
%     \item **Example Script:** `./scripts/launch_sim.sh`
%     \item **World Configuration File:** `04_Sim/ghost_sim/worlds/default.world`
%     \item **API Documentation:** [Link to relevant API or repo]
% \end{itemize}

\subsection*{Findings}
% Summarize the key things you learned or discovered.
\begin{itemize}
    \item LemLib supports odometry that tracks the position of the robot
    \item PID tuning can be used to get accurate measurements
    \item LemLib also supports pure pursuit which can be used to create complex paths for the robot to follow
\end{itemize}
% Example:
% \begin{itemize}
%     \item The simulation environment is stored in the `default.world` file in the `04_Sim/ghost_sim/worlds` directory.
%     \item The simulation environment can store information about the world, models, and other simulation parameters.
%     \item The simulation environment can be interacted with using the Gazebo GUI and the `gz` command-line tool.
%     \item GHOST has setup a controller interface for the robot in the `ghost_control` package.
% \end{itemize}

\subsection*{Issues}
% Document any blockers, bugs, or unresolved questions. Include error messages if relevant.
\begin{itemize}
    \item Unable to get TrackingWheel values to read correctly
\end{itemize}
% Example:
% \begin{itemize}
%    \item [Gazebo GUI Not Responding] – Tried restarting the simulation environment and the GUI, but the issue persisted.
%    \item [Error: Unable to Load Model] – Received an error message when trying to load a custom model into the simulation environment.
%    \item [ROS2 Environment Not Found] – The ROS2 environment was not found when trying to launch the simulation.
% \end{itemize}

\subsection*{Code Snippets / Commands Used}
% Include any important code snippets, shell commands, or configurations that were part of the exploration.
\begin{lstlisting}
     
\end{lstlisting}
% Example:
% \textbf{Launch Simulation Environment:}
% \begin{lstlisting}[language=bash]
% # Example command to launch the simulation environment
% ./scripts/launch_sim.sh
% \end{lstlisting}
% \textbf{Graph Plotting in Python:}
% \begin{lstlisting}[language=python]
%     # Example Python code snippet
%     import numpy as np
%     import matplotlib.pyplot as plt

%     x = np.linspace(0, 10, 100)
%     y = np.sin(x)

%     plt.plot(x, y)
%     plt.show()
% \end{lstlisting}