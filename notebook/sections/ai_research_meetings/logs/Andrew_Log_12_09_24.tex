\section{LemLib and CasADi Path Planning}
\timestamp{Andrew Needham}{[12/9/24]}{}

\subsection{Goals}
\begin{itemize}
    \item Update the solver to avoid the middle of the field
    \item Add option for more obstacles to avoid
    \item Make sure the solver outputting valid paths for LemLib
    \item Update the cost function to minimize the amount of time
\end{itemize}

\subsection{Methods}
\begin{itemize}
    \item Review the existing code to determine what needs to be changed and what can be removed
    \item Test the program on existing VEX drivetrain with the LemLib path program
\end{itemize}

\subsection{Results}
\begin{itemize}
    \item The solver splits the path to avoid the middle of the field
    \item The path output is able to be used for LemLib however the speed is not being used correctly
    \item The solver now minimizes the amount of time and dynamically changes the time step
    \item GitHub repo for CasADi test program: \href{https://github.com/msoe-vex/CasADi-Test.git}{https://github.com/msoe-vex/CasADi-Test.git}
    \item GitHub repo for another users implementation: \href{https://github.com/tianchenji/path-planning.git}{https://github.com/tianchenji/path-planning.git}
\end{itemize}

\subsection{Action Items}
\begin{itemize}
    \item Need to make sure the LemLib max speed parameter is being used correctly so that the path run properly
    \item Possibly adjust the hit-box of the robot to make sure the shape of the robot is being used correctly
    \item After that, will possibly work on converting the Python program to a C++ implementation
\end{itemize}