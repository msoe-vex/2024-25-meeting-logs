\section*{Powering Jetson Orin Nano Over Battery}
\timestamp{Evan Roegner}{[11/14/24]}{}

\subsection*{Objective / Focus}
% Briefly describe the goal of this topic. Try to explain what you expolored in the topic you chose.
% Example: The goal of this exploration was to understand the ROS2 environment and how to set up a simulation environment for the GHOST project
The goal for this was figure out a method of powering the Jetson Orin Nano off of a 5V USB Power bank

\subsection*{Questions Explored}
% List the specific questions or areas you focused on. You can paste the questions you have in the exploration topic and add any you found during the exploration.
\begin{itemize}
    \item What ports on the nano accept power?
    \item What connectors or cables are needed?
    \item How much power does the Nano need?
    \item How can USB-A/USB-C power out be converted to a standard that the nano accepts?
\end{itemize}
% Example:
% \begin{itemize}
%     \item What information can the sim env store?
%     \item What is the filetype of the world and the model files?
%     \item How can we interact with the simulation environment?
% \end{itemize}

\subsection*{Steps Taken / Methodology}
% Document the exact steps you followed, including paths to relevant files and commands used. If you tried several approaches, list them as separate lists.
\begin{enumerate}
    \item Nano usb-c plugged into 15V=3A power
    \item Watched NVIDIA Jetson on Battery Power video
    \item Searched for 5V to 19V step-up converter
    \item Looked into potentially using POE
\end{enumerate}
% Example:
% \testbf{Exploration Steps:}
% \begin{enumerate}
%     \item Setup ROS2 environment following the instructions in the GHOST project documentation.
%     \item Launch the simulation environment using the script `launch_sim.sh` in the `scripts` directory.
%     \item Explore the simulation environment in the "default.world" file in the `04_Sim/ghost_sim/worlds` directory and the "models" menu in the Gazebo GUI.
% \end{enumerate}

\subsection*{Findings / Insights}
% Summarize the key things you learned or discovered.
\begin{itemize}
    \item Our battery outputs at 5V, but the nano needs 9V-19V
    \item We can convert between voltages with a step-up converter
    \item Most step-up converters on the market that convert from 5V usb to 9V-19V barrel only provide up to 10 watts, which is not enough
    \item If we get an expansion board, Power over Ethernet is possible, but likely impractical
    \item Absolute minimum amount of power would be 15 Watts, but more is recommended
    \item Inner barrel positive, outer is negative
\end{itemize}
% Example:
% \begin{itemize}
%     \item The simulation environment is stored in the `default.world` file in the `04_Sim/ghost_sim/worlds` directory.
%     \item The simulation environment can store information about the world, models, and other simulation parameters.
%     \item The simulation environment can be interacted with using the Gazebo GUI and the `gz` command-line tool.
%     \item GHOST has setup a controller interface for the robot in the `ghost_control` package.
% \end{itemize}

\subsection*{Items needed}
\begin{itemize}
    \item 5V USB -> 9V-19V 5.5mmx2.5mm Barrel, >= 20 Watt output
    \item Alternatively, generic 5V->9V-19V step-up adaptor with correct connections soldered on would work
\end{itemize}

\subsection*{Challenges / Issues}
% Document any blockers, bugs, or unresolved questions. Include error messages if relevant.
\begin{itemize}
    \item Need to find an applicable power adaptor, or potentially create our own with a generic step-up adapter.
\end{itemize}
% Example:
% \begin{itemize}
%    \item [Gazebo GUI Not Responding] – Tried restarting the simulation environment and the GUI, but the issue persisted.
%    \item [Error: Unable to Load Model] – Received an error message when trying to load a custom model into the simulation environment.
%    \item [ROS2 Environment Not Found] – The ROS2 environment was not found when trying to launch the simulation.
% \end{itemize}

% Example:
% \textbf{Launch Simulation Environment:}
% \begin{lstlisting}[language=bash]
% # Example command to launch the simulation environment
% ./scripts/launch_sim.sh
% \end{lstlisting}
% \textbf{Graph Plotting in Python:}
% \begin{lstlisting}[language=python]
%     # Example Python code snippet
%     import numpy as np
%     import matplotlib.pyplot as plt

%     x = np.linspace(0, 10, 100)
%     y = np.sin(x)

%     plt.plot(x, y)
%     plt.show()
% \end{lstlisting}