\section*{CasADi and GHOST Path planning}
\timestamp{Andrew Needham}{[10/29/24]}{}

\subsection*{Research CasADi library and GHOST implementation of path planning}
The goal for this exploration is to determine how to use the CasADi library to generate path trajectories for the robot
% Briefly describe the goal of this topic. Try to explain what you expolored in the topic you chose.
% Example: The goal of this exploration was to understand the ROS2 environment and how to set up a simulation environment for the GHOST project.

\subsection*{Questions Explored}
% List the specific questions or areas you focused on. You can paste the questions you have in the exploration topic and add any you found during the exploration.
\begin{itemize}
    \item What is CasADi and what does it do?
    \item How does GHOST do path planning?
    \item What are the inputs and outputs of the algorithm?
    \item How can CasADi be used to generate path trajectories
\end{itemize}
% Example:
% \begin{itemize}
%     \item What information can the sim env store?
%     \item What is the filetype of the world and the model files?
%     \item How can we interact with the simulation environment?
% \end{itemize}

\subsection*{Methodology}
% Document the exact steps you followed, including paths to relevant files and commands used. If you tried several approaches, list them as separate lists.
\begin{enumerate}
    \item Reading through CasADi documentation
    \item Checking GHOST public repo for CasADi implementation and path planning code
    \item Create a test program to generate path and become familiar with the library
\end{enumerate}
% Example:
% \testbf{Exploration Steps:}
% \begin{enumerate}
%     \item Setup ROS2 environment following the instructions in the GHOST project documentation.
%     \item Launch the simulation environment using the script `launch_sim.sh` in the `scripts` directory.
%     \item Explore the simulation environment in the "default.world" file in the `04_Sim/ghost_sim/worlds` directory and the "models" menu in the Gazebo GUI.
% \end{enumerate}

\subsection*{Code Locations}
% Provide paths to relevant scripts, configuration files, or datasets used. If you used a script, provide the path to the script and the command used to run it.
\begin{itemize}
    \item GHOST VEXU Repository: \href{https://github.com/VEXU-GHOST/VEXU_GHOST}{link}
    \item Swerve drive trajectory generation \href{https://github.com/VEXU-GHOST/VEXU_GHOST/blob/develop/11_Robots/ghost_swerve_mpc_planner/src/casadi_swerve_model_generation.cpp}{link}
    \item Test program repo: \href{https://github.com/satsinush/CasADi-Test.git}{link}
\end{itemize}
% Example:
% \begin{itemize}
%     \item **Example Script:** `./scripts/launch_sim.sh`
%     \item **World Configuration File:** `04_Sim/ghost_sim/worlds/default.world`
%     \item **API Documentation:** [Link to relevant API or repo]
% \end{itemize}

\subsection*{Findings}
% Summarize the key things you learned or discovered.
\begin{itemize}
    \item CasADi is a general purpose library with the focus of solving optimzation problems given a set of constraints
    \item Inputs of CasADi is a function of multiple variables and a number of constraints for the function
    \item The GHOST sytems uses serialized messaging to communicate between different classes and parts of the robot
\end{itemize}
% Example:
% \begin{itemize}
%     \item The simulation environment is stored in the `default.world` file in the `04_Sim/ghost_sim/worlds` directory.
%     \item The simulation environment can store information about the world, models, and other simulation parameters.
%     \item The simulation environment can be interacted with using the Gazebo GUI and the `gz` command-line tool.
%     \item GHOST has setup a controller interface for the robot in the `ghost_control` package.
% \end{itemize}

\subsection*{Issues}
% Document any blockers, bugs, or unresolved questions. Include error messages if relevant.
\begin{itemize}
    \item Unable to locate where in the GHOST repo CasADi was implmented for a tank drive
    \item Unable to generate a path that minimizes the time taken to get to the target point
\end{itemize}
% Example:
% \begin{itemize}
%    \item [Gazebo GUI Not Responding] – Tried restarting the simulation environment and the GUI, but the issue persisted.
%    \item [Error: Unable to Load Model] – Received an error message when trying to load a custom model into the simulation environment.
%    \item [ROS2 Environment Not Found] – The ROS2 environment was not found when trying to launch the simulation.
% \end{itemize}

\subsection*{Code Snippets / Commands Used}
% Include any important code snippets, shell commands, or configurations that were part of the exploration.
\begin{lstlisting}
     
\end{lstlisting}
% Example:
% \textbf{Launch Simulation Environment:}
% \begin{lstlisting}[language=bash]
% # Example command to launch the simulation environment
% ./scripts/launch_sim.sh
% \end{lstlisting}
% \textbf{Graph Plotting in Python:}
% \begin{lstlisting}[language=python]
%     # Example Python code snippet
%     import numpy as np
%     import matplotlib.pyplot as plt

%     x = np.linspace(0, 10, 100)
%     y = np.sin(x)

%     plt.plot(x, y)
%     plt.show()
% \end{lstlisting}